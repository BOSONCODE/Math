\documentclass{article}
\usepackage[space]{ctex}
\usepackage{amsmath}
\usepackage[CJKbookmarks]{hyperref}
\hypersetup{
colorlinks=true,
linkcolor=black
} %���Ŀ¼��������
\renewcommand{\contentsname}{\centerline{Ŀ¼}}
\title{Positive-definite and positive semi-definite matrix}
\date{}
\begin{document}
\maketitle
\section{Notations}
\begin{itemize}
    \item the conjugate transpose of matrix A is denoted as $A^H$
    \item the transpose of matrix A is denoted as $A^T$
    \item the conjugate of matrix A is denoted as $A^*$
\end{itemize}
\section{Definition}
    \subsection{Hermitian matrix}
        In mathematics, a Hermitian matrix (or self-adjoint matrix) is a complex square matrix that is equal to its own conjugate transpose��that is, the element in the i-th row and j-th column is equal to the complex conjugate of the element in the j-th row and i-th column, for all indices i and j: \newline
        $A = A^H$ \newline
        Hermitian matrices can be understood as the complex extension of real symmetric matrices. \newline
       
\section{properties}
    \begin{itemize}
        \item  a Hermitian matrix's diagonal elements must be real, as they must be their own complex conjugate
        \item  if a square matrix A equals the multiplication of a matrix and its conjugate transpose, that is, $A = BB^H$, then A is a Hermitian positive semi-definite matrix.Furthermore, if B is row full-rank, then A is positive definite.
        \item for an arbitrary complex valued vector v the product $v^HAv$ is real because of $v^HAv = (v^HAv)^H$
    \end{itemize}
\section{How to judge a positive-definite matrix}
    \begin{itemize}
        \item a matrix is positive definite if it's symmetric and all its pivots are positive. Just perform elimination and examine the diagonal terms.
            \begin{gather*}
                \begin{pmatrix}
                    1 & 2 \\
                    2 & 1
                \end{pmatrix}
            \end{gather*}
            after performing the elimination:
            \begin{gather*}
                \begin{pmatrix}
                    1 & 2 \\
                    0 & -3
                \end{pmatrix}
            \end{gather*}
            the pivots are 1 and -3. -3 is not positive, so the matrix is not positive definite matrix.
        \item the k-th pivot of a matrix is $d_k = \frac{det(A_k)}{det(A_{k-1})}$ where $A_k$ is the upper left kxk submatrix. All the pivots will be positive if and only if $det(A_k) \geq 0$  for all $1 \leq k \leq  n$
            \begin{gather*}
                \begin{pmatrix}
                    2 & -1 & 0 \\
                    -1 & 2 & -1 \\
                    0 & -1 & 2
                \end{pmatrix}
            \end{gather*}
            \begin{gather*}
                \begin{vmatrix}
                    2
                \end{vmatrix} = 2,
                \begin{vmatrix}
                    2 & -1 \\
                    -1 & 2
                \end{vmatrix} = 3,
                \begin{vmatrix}
                    2 & -1 & 0 \\
                    -1 & 2 & -1 \\
                    0 & -1 & 2
                \end{vmatrix} = 4
            \end{gather*}
            $2, 3, 4 \geq 0$ so the matrix is the positive definite matrix
    \end{itemize}
\end{document} 